% ############################################################ DOCUMENT CLASS #

\documentclass[ 8pt, c5paper, twoside, openright]{extbook}

%\documentclass[10pt, a4paper, twoside, openright]{extbook}
%\documentclass[ 8pt, B5paper, twoside, openright]{extbook} 


% ########################################################### PAGE GEOMETRIES #

\usepackage%[showframe]
           {geometry}

% Basic geometry of document
\geometry
  { headsep    =   \baselineskip
  , textwidth  = 42\baselineskip
  , textheight = 60\baselineskip
  , hmarginratio = 2:3
  , vmarginratio = 2:3
  , bindingoffset = 0cm
  , onecolumn
  }

% Geometry of the dictionary section
\newcommand\dictionarygeometry{\newgeometry
  { textwidth  = 42\baselineskip
  , textheight = 60\baselineskip
  , hmarginratio = 2:3
  , vmarginratio = 3:2
  , bindingoffset = 0cm
  , twocolumn
  }}


% ################################################################# LANGUAGES #

\usepackage[icelandic, latin, czech]{babel}
\usepackage{csquotes}


% ################################################################## ENCODING #

\usepackage[utf8]{inputenc}

% Smashing uppercase letters with diacritics
% ------------------ THIS IS EVIL AND SHOULD NEVER BE DONE -------------------
% Don't smash anything if we are on the first run to produce the index
\IfFileExists{figures.idx}{
  % Czech uppercase letters with diacritics:
  \DeclareUnicodeCharacter{00C1}{\strut\smash{\' A}} % Á
  \DeclareUnicodeCharacter{010C}{\strut\smash{\v C}} % Č
  \DeclareUnicodeCharacter{010E}{\strut\smash{\v D}} % Ď
  \DeclareUnicodeCharacter{00C9}{\strut\smash{\' E}} % É
  \DeclareUnicodeCharacter{011A}{\strut\smash{\v E}} % Ě
  \DeclareUnicodeCharacter{00CD}{\strut\smash{\' I}} % Í
  \DeclareUnicodeCharacter{0147}{\strut\smash{\v N}} % Ň
  \DeclareUnicodeCharacter{00D3}{\strut\smash{\' O}} % Ó
  \DeclareUnicodeCharacter{0158}{\strut\smash{\v R}} % Ř
  \DeclareUnicodeCharacter{0160}{\strut\smash{\v S}} % Š
  \DeclareUnicodeCharacter{0164}{\strut\smash{\v T}} % Ť
  \DeclareUnicodeCharacter{00DA}{\strut\smash{\' U}} % Ú
  \DeclareUnicodeCharacter{016E}{\strut\smash{\r U}} % Ů
  \DeclareUnicodeCharacter{00DD}{\strut\smash{\' Y}} % Ý
  \DeclareUnicodeCharacter{017D}{\strut\smash{\v Z}} % Ž
  % Icelandic uppercase letters with diacritics: ÁÉÍÓÚÝ (ÐÞÆ) and
  \DeclareUnicodeCharacter{00D6}{\strut\smash{\" O}} % Ö
}{}
% ------------------ THAT WAS EVIL AND SHOULD NEVER BE DONE ------------------

% Non-breakable space
\DeclareUnicodeCharacter{00A0}{~}

% Non-breakable hyphen
\newcommand*{\nobreakhyphen}{\mbox{-}}
\DeclareUnicodeCharacter{2011}{\nobreakhyphen}


% ############################################################ FONTS, SYMBOLS #

\usepackage[T1]{fontenc}

\usepackage{tgpagella}
\usepackage[scaled=0.88]{helvet}      % relative scale of the two fonts

\def\phvfamily{\fontfamily{phv}\selectfont}

%\usepackage{stmaryrd}
% Let's keep this minimal instead of using the package:
\DeclareSymbolFont{stmry}{U}{stmry}{m}{n}
%\SetSymbolFont{stmry}{bold}{U}{stmry}{b}{n}
\DeclareMathSymbol\shortrightarrow\mathrel{stmry}{"01}
\DeclareMathSymbol\shortuparrow\mathrel{stmry}{"02}

% Methinks these are not necessary
%\usepackage{dingbat}
%\usepackage{manfnt}
%\usepackage{latexsym}

\usepackage{pifont}


% #################################################################### LAYOUT #

\def\HUGE {\fontsize{23.887872pt}{3\baselineskip}\selectfont}
\def\Huge {\fontsize{19.906560pt}{3\baselineskip}\selectfont}
\def\huge {\fontsize{16.588800pt}{3\baselineskip}\selectfont}
\def\LARGE{\fontsize{13.824000pt}{2\baselineskip}\selectfont}
\def\Large{\fontsize{11.520000pt}{2\baselineskip}\selectfont}
\def\large{\fontsize{ 9.600000pt}{2\baselineskip}\selectfont}
% \normalsize is {8pt}{9.6pt}

% Two columns layout
\setlength\columnsep    {2\baselineskip}
\setlength\columnseprule{0.4pt}

% Necessary for baseline alignment
\topskip=\baselineskip
\raggedbottom
\setlength\parskip{0pt} % it's better to avoid glue

% Temporarily suppress warnings
\hbadness=10000
\vbadness=10000

% This value is ENORMOUS but I can't think of a better non-manual solution.
% Stil, microtype helps a lot to lower it.
\setlength\emergencystretch{17pt}

%\renewcommand\baselinestretch{1.0}
%\setlength\baselineskip{9.6pt}


% ######################################################### ASSORTED PACKAGES #

\usepackage[rgb]{xcolor}

\usepackage{tikz}
\usetikzlibrary{calc}

\usepackage{tipa}                      % IPA phonetics

%\usepackage{fourier-orns}              % ornaments
%\usepackage{amsmath}                   % non-breakable dash
%\usepackage{hyphenat}                  % no hyphen in abbreviations

\usepackage{hanging}
\usepackage{fix2col}
\usepackage{fixltx2e}                  % subscript
\usepackage{pagecolor}
\usepackage{afterpage}                 % Used in coloring trick for covers
\usepackage{pdfpages}

\usepackage{multirow}
\usepackage{multicol}
\usepackage{rotating}

\usepackage{etoolbox}

\usepackage[ISBN=978-80-260-2385-2,SC0]{ean13isbn}

\usepackage{pgffor}


% ############################################################ BASELINE GRID #

\usepackage{eso-pic}
\usepackage{tikzpagenodes}

% Command to draw a baseline grid
\newcommand\drawbaselinegrid{%
  \begin{tikzpicture}[overlay,remember picture]
    \draw [red!30!white, ultra thin, dashed]
      (0,0) grid [ ystep  = \baselineskip, xstep = \textwidth
                 , shift  = (current page text area.north west)
                 , yshift = -\dp\strutbox
                 ] ++(\textwidth,-\textheight);
    \draw [red!30!white, thin]
      (0,0) grid [ step = \baselineskip, xstep = \textwidth
                 , shift=(current page text area.north)
                 ] ++(0.5\textwidth,-\textheight)
      (0,0) grid [ step = \baselineskip, xstep = \textwidth
                 , shift=(current page text area.north)
                 ] ++(-0.5\textwidth,-\textheight);
    \draw[black!10!white]
      (current page text area.north west)
        rectangle (current page text area.south east);
  \end{tikzpicture}}

% Draw a baseline grid on every page
\AddToShipoutPicture{\drawbaselinegrid}


% ################################################################## METADATA #

% This is not actually being used. ------------------------------------------ !

%\title{%
%  \textbf{Islandsko-český studijní slovník}%
%  \thanks{Tato kniha byla vytvořena v \LaTeX{}u pod Ubuntu 14.04.%
%    Poděkování patří všem autorům, kteří publikují pod svobodnými licencemi.}}
%\author{Aleš Chejn, Ján Zaťko, Jón Gíslason}
%\date{říjen 2011}


% ############################################################## BIBLIOGRAPHY #

\usepackage[style=authoryear]{biblatex}
\addbibresource{slovnik.bib}


% ##################################################################### INDEX #

\usepackage{imakeidx}

% Index of authors of photographs
\makeindex
  [ name = figures
  , title = {Seznam autorů fotografií a ilustrací}
  , intoc
  , columnseprule
  , columnsep = \columnsep
  , noautomatic                       % We handle this in make.sh
  ]

% icelandic rules start
\usepackage{filecontents}
\def\mygroup#1{{\bfseries#1}}
\begin{filecontents*}{style.xdy}
;; style.xdy
(markup-letter-group :open-head "~n\mygroup{" :close-head "}")
\end{filecontents*}
% icelandic rules end


% ##################################################################### LISTS #

\usepackage{expdlist}
\usepackage{enumitem}

\setlist{nosep, topsep=\baselineskip}
\setlist[itemize]{leftmargin=*,label=\scalebox{.6}{\textbullet}}


% #################################################################### FLOATS #

\usepackage{caption}
\captionsetup{labelformat=empty}

% Alter some LaTeX defaults for better treatment of figures:
% See p.105 of "TeX Unbound" for suggested values. 
% See pp. 199-200 of Lamport's "LaTeX" book for details.
% ------------------------------------- Parameters for ALL pages:
\renewcommand \topfraction    {0.9}   %   max fraction of floats at top
\renewcommand \bottomfraction {0.8}   %   max fraction of floats at bottom
% ------------------------------------- Parameters for TEXT (non float) pages:
\setcounter{topnumber}    {1}
\setcounter{bottomnumber} {1}
\setcounter{totalnumber}  {1}         %   2 may work better
\setcounter{dbltopnumber} {2}         %   for 2-column pages
\renewcommand \dbltopfraction   {0.9} %   fit big float above 2-col. text
\renewcommand \textfraction     {0.07}%   allow minimal text w. figs
% ------------------------------------- Parameters for FLOAT (non text) pages:
\renewcommand \floatpagefraction{0.7} %   require fuller float pages
% N.B.: floatpagefraction MUST be less than topfraction !!
\renewcommand{\dblfloatpagefraction}{0.7} % require fuller float pages
% remember to use [htp] or [htpb] for placement

\makeatletter
\setlength{\@fptop}{0pt}
\setlength{\@fpbot}{0pt plus 1fil}
\makeatother

% This kills spacing around floats
\setlength \intextsep        {0pt}
\setlength \floatsep         {0pt}
\setlength \textfloatsep     {0pt}
\setlength \dbltextfloatsep  {0pt}
\setlength \dblfloatsep      {0pt}

% This kills spacing around captions
\setlength \abovecaptionskip {0pt}
\setlength \belowcaptionskip {0pt}


% #################################################################### TABLES #

\usepackage{booktabs}
\usepackage{tabularx}
\usepackage{longtable}
\usepackage{ltxtable}

\newcommand\LTXfw[1]{\LTXtable{\columnwidth}{#1}}

% ------------------------------------------------- I HAVE NO CLUE WHAT THIS IS
\begingroup
  \makeatletter
  \catcode`\-=\active
  \AtBeginDocument{
  \def\@@@cmidrule[#1-#2]#3#4{\global\@cmidla#1\relax
    \global\advance\@cmidla\m@ne
    \ifnum\@cmidla>0\global\let\@gtempa\@cmidrulea\else
    \global\let\@gtempa\@cmidruleb\fi
    \global\@cmidlb#2\relax
    \global\advance\@cmidlb-\@cmidla
    \global\@thisrulewidth=#3
    \@setrulekerning{#4}
    \ifnum\@lastruleclass=\z@\vskip \aboverulesep\fi
    \ifnum0=`{\fi}\@gtempa
    \noalign{\ifnum0=`}\fi\futurenonspacelet\@tempa\@xcmidrule}
  }
\endgroup

% These values ensure that for top, mid and bottom rules
% space above + rule width + space below = baselineskip
\setlength \abovetopsep    {0.43\baselineskip}
\setlength \heavyrulewidth {0.10\baselineskip}
\setlength \belowbottomsep {0.43\baselineskip}
\setlength \aboverulesep   {0.47\baselineskip}
\setlength \lightrulewidth {0.06\baselineskip}
\setlength \belowrulesep   {0.47\baselineskip}

%%% % declention and conjugation tables
%%% \usepackage{floatrow}
%%% \DeclareFloatFont{footnotesize}{\footnotesize}
%%% % "scriptsize" is defined by floatrow, "tiny" not
%%% \floatsetup[table]{font=footnotesize}


% ################################################################### FIGURES #

\usepackage{adjustbox} % loads also graphicx

\graphicspath{%
  {/home/chejnik/Dokumenty/hvalur.org/images/biolib/full/}%
  {/home/chejnik/Dokumenty/hvalur.org/images/uploaded_files/}}

\newlength\dicfigheight
\newlength\dicfigskip

\newcommand\dictionaryfigure[3]{ % {filename}{caption}{index entry}
  \setlength\fboxsep{0pt}\setlength\fboxrule{0.5pt}
  \def\dicfigbox{\fbox{%
    \adjincludegraphics [ max height = 0.8\columnwidth
                        , max width  = 0.8\columnwidth ] {#1}}}
  \setlength\abovecaptionskip {\dp\strutbox}
  \setlength\belowcaptionskip {\ht\strutbox} 
  \setlength\dicfigheight {\heightof\dicfigbox+\fboxrule}
  \setlength\dicfigskip
    {\baselineskip*\numexpr1+\dicfigheight/\baselineskip\relax-\dicfigheight}
  \begin{figure}[hbt]
    \centering\vspace*{\dicfigskip}\dicfigbox\caption{#2}\index[figures]{#3}
  \end{figure}
}

\ifdummyfigures
  \let\olddicFigure\dictionaryfigure
  \renewcommand\dictionaryfigure[3]{\olddicFigure{example-image-a}{#2}{#3}}
\fi


% #################################################################### COLORS #

%\usepackage{color}

% http://www.colorschemer.com/online.html
% main color - grammar color
% #66023C
\definecolor {darkgreen}          {rgb} {0.40, 0.01, 0.24}

%\definecolor {royalazure}         {rgb} {0.00, 0.22, 0.66}
%\definecolor {brown}              {rgb} {0.40, 0.01, 0.24}

% COLORS FOR THUMB INDEXES
%\definecolor {darkblue}           {rgb}{0.00, 0.00, 0.55} 
%\definecolor {upforestgreen}      {rgb}{0.00, 0.27, 0.13} 
%\definecolor {islamicgreen}       {rgb}{0.00, 0.56, 0.00} 
%\definecolor {blue(ryb)}          {rgb}{0.01, 0.28, 1.00} 
%\definecolor {limegreen}          {rgb}{0.20, 0.80, 0.20} 
%\definecolor {unitednationsblue}  {rgb}{0.36, 0.57, 0.90} 
%\definecolor {screamingreen}      {rgb}{0.46, 1.00, 0.44} 
%\definecolor {patriarch}          {rgb}{0.50, 0.00, 0.50} 
%\definecolor {oucrimsonred}       {rgb}{0.60, 0.00, 0.00} 
%\definecolor {babyblueeyes}       {rgb}{0.63, 0.79, 0.95} 
%\definecolor {vividcerise}        {rgb}{0.85, 0.11, 0.51} 
%\definecolor {orchid}             {rgb}{0.85, 0.44, 0.84} 
%\definecolor {icterine}           {rgb}{0.99, 0.97, 0.37} 
%\definecolor {orange-red}         {rgb}{1.00, 0.27, 0.00} 
%\definecolor {orange(colorwheel)} {rgb}{1.00, 0.50, 0.00} 
%\definecolor {cottoncandy}        {rgb}{1.00, 0.74, 0.85} 

\definecolor {title}              {RGB} {  16,   13,   32}
%\definecolor {golden}             {RGB} { 241,  184,   45}
%\definecolor {freqstar}           {RGB} {  91,    3,   99}
\definecolor {newdev}             {RGB} {   3,   11,   99}


%\definecolor {color1}             {RGB} { 239,  222,  205}
%\definecolor {color2}             {RGB} { 251,  206,  177}
%\definecolor {color3}             {RGB} { 251,  185,  142}
%\definecolor {color4}             {RGB} { 248,  166,  112}
%\definecolor {color5}             {RGB} { 216,  231,  246}
%\definecolor {color6}             {RGB} { 186,  215,  244}
%\definecolor {color7}             {RGB} { 152,  197,  243}
%\definecolor {color8}             {RGB} { 118,  181,  245}
%\definecolor {color9}             {RGB} { 181,  249,  185}
%\definecolor {color10}            {RGB} { 160,  248,  167}
%\definecolor {color11}            {RGB} { 128,  247,  136}
%\definecolor {color12}            {RGB} {  87,  248,   98}
%\definecolor {color13}            {RGB} { 250,  241,  138}
%\definecolor {color14}            {RGB} { 250,  240,  111}
%\definecolor {color15}            {RGB} { 244,  239,   85}
%\definecolor {color16}            {RGB} { 245,  239,   57}

%\definecolor {lightgrey}          {RGB} { 105,  105,  105}
%\definecolor {darkgreen_real}     {rgb} {0.00, 0.50, 0.00}









% ############################################################# THUMB INDEXES #

\def\alphabet
  {a,á,b,c,d,e,é,f,g,h,i%
  ,í,j,k,l,m,n,o,ó,p,r,s%
  ,t,u,ú,v,w,x,y,ý,þ,æ,ö}

\def\ALPHABET
  {A,Á,B,C,D,E,É,F,G,H,I%
  ,Í,J,K,L,M,N,O,Ó,P,R,S%
  ,T,U,Ú,V,W,X,Y,Ý,Þ,Æ,Ö}


%aábcd(ð)eéfghi
%íjklmnoóp(q)rs
%tuúvwxyýþæö

\definecolorseries{thumbcolors}{hsb}{grad}[hsb]{0,0.9,0.7}[hsb]{-1,0,0}
\resetcolorseries[33]{thumbcolors}

\definecolorset{Hsb}{thumbcolor}{}
  {1,    1, .8, .7
  ;2,   28, .8, .7
  ;3,   46, .8, .7
  ;4,   56, .8, .7
  ;5,   76, .8, .7
  ;6,  130, .8, .7
  ;7,  153, .8, .7
  ;8,  174, .8, .7
  ;9,  196, .8, .7
  ;10, 211, .8, .7
  ;11, 230, .8, .7}


\definecolor{color13}{RGB}{ 205,194, 18}  
\definecolor{color14}{RGB}{ 204,162, 24}  
\definecolor{color1} {RGB}{ 182, 86,  0}  
\definecolor{color2} {RGB}{ 143,  9,  6}  
\definecolor{color3} {RGB}{   3, 23,118}  
\definecolor{color4} {RGB}{   0, 82,168}  
\definecolor{color5} {RGB}{   0, 85,142}  
\definecolor{color6} {RGB}{   0,115,162}  
\definecolor{color7} {RGB}{  34,146,186}  
\definecolor{color8} {RGB}{  40,159,153}  
\definecolor{color9} {RGB}{   0,125,111}  
\definecolor{color10}{RGB}{   4,107, 60}  
\definecolor{color11}{RGB}{  71,134, 81}  
\definecolor{color12}{RGB}{ 109,134, 42}  

\newcommand\BoxColor[1]{%
\ifcase#1\relax
\or thumbcolor1%
\or thumbcolor2%
\or thumbcolor3%
\or thumbcolor4%
\or thumbcolor5%
\or thumbcolor6%
\or thumbcolor7%
\or thumbcolor8%
\or thumbcolor9%
\or thumbcolor10%
\or thumbcolor11%
\else\BoxColor{\the\numexpr#1-11\relax}\fi}

\newcommand\OldBoxColor[1]{%
\ifcase#1\relax
\or color1\or color14\or color1\or color2\or color3\or color4\or color5%
\or color6\or color7\or color8\or color9\or color10\or color11\or color12%
\else\OldBoxColor{\the\numexpr#1-14\relax}\fi}

% THUMB INDEXES
% new counter to hold the current number of the letter to determine the vertical position
\newcounter{letternum}
% newcounter for the sum of all letters to get the right height of a box
%\newcounter{lettersum}
%\setcounter{lettersum}{34}
% some margin settings
\newlength\thumbheight \setlength\thumbheight {1.75\baselineskip}
\newlength\thumbwidth  \setlength\thumbwidth  {0.4\paperwidth-0.4\textwidth}
\newlength\thumbmargin \setlength\thumbmargin {0.5\paperheight-17\thumbheight}
% style the boxes
\tikzset{
  thumb/.style={
    text=white,
    text centered, 
    minimum height=\thumbheight,
    minimum width=\thumbwidth,
    outer sep=0pt,
    font=\sffamily\bfseries\large,
 }}

\usepackage{everypage}


\def\thumblast{}  % Last thumb of previous page
\def\thumbstack{} % Stack holding thumbs of current page
% At start of page, if there are no new thumbs then use the old one
\AddEverypageHook{\if\relax\thumbstack\relax\xdef\thumbstack{\thumblast}\fi}
\def\thumbpush#1{\xdef\thumbstack{#1,\thumbstack}}

\newcommand{\drawthumb}[3]{%
  \begin{tikzpicture}[remember picture, overlay]
    \node [thumb, fill=\BoxColor{#1}, anchor=south #2]
       at ($(current page.north #2)-(0,\thumbmargin+#1\thumbheight)$) {#3};
  \end{tikzpicture}}

\newcommand\drawrectothumb[1]{\drawthumb{#1}{east}{\csname Let#1\endcsname}}
\newcommand\drawversothumb[1]{\drawthumb{#1}{west}{\csname Let#1\endcsname}}

\def\dorectothumbs{\expandafter\thumbgobblewith\thumbstack<\drawrectothumb>}
\def\doversothumbs{\expandafter\thumbgobblewith\thumbstack<\drawversothumb>}

\def\thumbgobblewith#1,#2<#3>{%
  \if\relax#1\relax\else#3{#1}\fi%
  \if\relax#2\relax\else\thumbgobblewith#2<#3>\fi%
  \gdef\thumblast{#1,}%
  \gdef\thumbstack{}}


% Add a thumb to the stack
\newcommand{\lettergroup}[1]{%
  % step the counter of the letters
  \refstepcounter{letternum}%
  \expandafter\gdef\csname Let\theletternum\endcsname{#1}%
  \thumbpush\theletternum
}


% ############################################################### PAGE STYLES #

\usepackage{fancyhdr}

% -------------------------------------- BASIC PAGE STYLE
\fancypagestyle{plain}{
  \fancyhf\relax                       % Clear header/footer
  \renewcommand \headrulewidth {0.0pt} % No header rule
  \renewcommand \footrulewidth {0.0pt} % No footer rule
  \fancyfoot[C]{\thepage}              % Page number in footer, centred
}

% -------------------------------------- DICTIONARY PAGE STYLE
\fancypagestyle{myheadings}{
  \fancyhf\relax                       % Clear header/footer
  \renewcommand \headrulewidth {0.4pt} % Thin header rule
  %\fancyhead[CO,CE]{\thepage}          % Page number in header, centred
  \fancyhead[CO]{\thepage\dorectothumbs}
  \fancyhead[CE]{\thepage\doversothumbs}
  \fancyhead[LE,LO]{\phvfamily\bfseries\rightmark}
  \fancyhead[RE,RO]{\phvfamily\bfseries\leftmark}
}

\pagestyle{plain}  


% ############################################################## LOCALIZATION #

% renames the index name
%\addto\captionsczech{%
%  \renewcommand\indexname{{Seznam autorů fotografií a ilustrací}}}

% renames the contents name
\addto\captionsczech{%
  \renewcommand\contentsname{Obsah}}

\renewcommand \contentsname {Obsah}
%\renewcommand \chaptername  {Kapitola}


% ################################################################ SECTIONING #
\usepackage[explicit]{titlesec}

\setlength{\beforetitleunit}{\baselineskip}
\setlength{\aftertitleunit} {\baselineskip}

\titlespacing*{\chapter}   {0em}{*0}{*3} % Remember there's a topskip too
\titlespacing*{\section}   {0em}{*1}{*2}
\titlespacing*{\subsection}{0em}{*1}{*2}

\newcommand\ghostdrop[2]{%
  \raisebox{-#1\baselineskip}[0pt][0pt]{#2}}

\titleformat {\chapter} {} {} {0pt} % {#1}
  {\ghostdrop{1.0}{\parbox[t]{\columnwidth}
     {\LARGE\bfseries\centering\MakeUppercase{#1}}}}

\def\longchapterskip{\vspace{2\baselineskip}}
\def\longsectionskip{\vspace{2\baselineskip}}

\titleformat {\section} {\bfseries} {} {0pt}
  {\ghostdrop{1.0}{\parbox[t]{\columnwidth}
     {\Large\bfseries\centering#1}}}

\titleformat {\subsection} {\bfseries} {} {0pt}
  {\ghostdrop{1.0}{\hspace{1em}\parbox[t]{\columnwidth-1em}
     {\large\bfseries#1}}}


\setlength\multicolsep{0pt}
%\BeforeBeginEnvironment{multicols}{\vspace{-\dp\strutbox}}


% ######################################################## TABLE OF CONTENTS #
\usepackage{titletoc}

\titlecontents {chapter} [0pt] {\addvspace\baselineskip\bfseries}
  {} {} {\titlerule*[2\baselineskip]{}\contentspage}

\titlecontents {section} [\baselineskip] {} {} {}
  {\titlerule*[1\baselineskip]{.}\contentspage}


% #################################################### ASSORTED CUSTOM MACROS #

\def\textIS#1{\foreignlanguage{icelandic}{#1}}
\def\textCS#1{\foreignlanguage{czech}{#1}}
\def\textLA#1{\foreignlanguage{latin}{#1}}


\usepackage{pgffor}

\newcommand\blspace[1][1]{\vspace{#1\baselineskip}}



% ####################################################### DICTIONARY COMMANDS #

\usepackage{xparse}

\newcommand \dicsymFrequent  {\raisebox{-.2ex}{\color{darkgreen}\ding{167}}}
\newcommand \dicsymSee       {$\shortrightarrow$}
\newcommand \dicsymCompare   {$\shortuparrow$}
\newcommand \dicsymIdiom     {$\diamondsuit$}
\newcommand \dicsymExampleIS {$\triangleright$}
\newcommand \dicsymExampleCS {\guilsinglright}
\newcommand \dicsymProverb   {{\color{newdev}\ding{96}}}
\newcommand \dicsymPhoto
  {\includegraphics[keepaspectratio,height=0.65em]{photo_icon}}


\NewDocumentCommand\dicEntry{o}
  {\par\hangpara{0.25\baselineskip}{1}%
   \IfValueT{#1}{\markboth{#1}{#1}}%
   \ignorespaces}

\newcommand\dicSubEntry
  {\\\hspace*{-\hangindent}}

\NewDocumentCommand\dicTerm{mo}
  {{\phvfamily\bfseries\textIS{#1}}%
   \IfValueT{#2}{, {\phvfamily\bfseries\textIS{#2}}}}

\def\dicIPA#1{\textipa{[#1]}}

\NewDocumentCommand\dicPos{smo}
  {{%
%\mbox{
\color{darkgreen}\small{\bfseries#2}\IfValueT{#3}{\nobreak\textsubscript{#3}}
%}
}}

\NewDocumentCommand\dicFlx{smo}
  {{%
%\mbox{
\color{darkgreen}\footnotesize#2\IfValueT{#3}{\nobreak\textsubscript{#3}}
%}
}}

%\newcommand\dicEntry[1][]
%  {\par\hangpara{0.5\baselineskip}{1}%
%   \markboth{#1}{#1}}
%\ifx\relax#1\relax\markboth{#2}{#2}\else\markboth{#1}{#1}\fi%



%\newcommand\dicTermList[1]
%  {\foreach \t [count=\n] in {#1}{\ifnum\n=1\else, \fi\dicTerm{\t}}}



\newcommand\dicExampleCS[1]
  {\dicsymExampleCS~\textit{#1}}
\newcommand\dicExampleIS[1]
  {\dicsymExampleIS~\textIS{\textit{#1}}}
\NewDocumentCommand\dicLink{sm}
  {\IfBooleanF#1{\dicsymSee~}\dicTerm{#2}}
\NewDocumentCommand\dicSynonym{sm}
  {(\IfBooleanF#1{\dicsymSee~}\textIS{\textit{#2}})}
\NewDocumentCommand\dicAntonym{sm}
  {(\IfBooleanF#1{\dicsymCompare~}\textIS{\textit{#2}})}
\NewDocumentCommand\dicCompare{sm}
  {(\IfBooleanF#1{\dicsymCompare~}\textIS{\textit{#2}})}

\newcommand\dicPhraseIS[1]{\textIS{\bfseries#1}}
\NewDocumentCommand\dicIdiom{mo}
  {\dicSubEntry{\color{newdev}\dicTerm{#1} %
     \IfValueTF{#2}{+ \dicTerm{#2} \markboth{#1 + #2}{#1 + #2}}{\markboth{#1}{#1}}}\dicsymIdiom}
\newcommand\dicProverb
  {\dicSubEntry\dicsymProverb}



\newcommand\dicLangCat[1]{{\footnotesize#1}}
\newcommand\dicFieldCat[1]{{\footnotesize#1}}


\newcommand\dicDirectTranslationCS[1]{#1}
\newcommand\dicIndirectTranslationCS[1]{{\footnotesize#1}}


%\newcommand\entry[3][]
%  {\hangpara{0.5\baselineskip}{1}%
%   \ifx\relax#1\relax\markboth{#2}{#2}\else\markboth{#1}{#1}\fi%
%   \dicTerm{#2}\ #3%
%   \par}
%
%\newcommand\devision[1]
%  {\hspace*{-\hangindent}%
%   \textIS{\color{newdev}{\phvfamily #1}}}
%

\usepackage{placeins}

\newcommand*{\dicLetter}[2]{%
  \FloatBarrier\newpage\phantomsection\pdfbookmark[1]{#1}{#2}%
  \noindent\parbox[b][9\baselineskip][c]{\columnwidth}
    {\centering\HUGE\strut\smash{\MakeUppercase{#1}~\MakeLowercase{#1}}}%
      %\par\xdef\tpd{\the\prevdepth}
      %\par\prevdepth\tpd%
  \expandafter\lettergroup{\MakeUppercase{#1}}}
  
%tabulky
\newcommand{\specialcell}[2][l]{%
\begin{tabular}[#1]{@{}l@{}}#2\end{tabular}}


%\newcommand{\devisionguide}[1]{\hspace*{0.3cm}{{{{\foreignlanguage{icelandic}{\color{newdev}{\phvfamily{\textbf{#1}}}}}}}}}
%\newcommand{\devisionguideindent}[1]{\hspace*{-0.1cm}{{{{\foreignlanguage{icelandic}{\color{newdev}{\phvfamily{\textbf{#1}}}}}}}}}

%\newcommand\n{$n$\nobreakdash-\hspace{0pt}} %nonbreakable dash in grammar endings
%\def \nobreakseq {\nobreak \hskip 0pt \hbox} %nolinebreak in case like (-s, -)

% rule line
\newcommand{\HRule}{\rule{\linewidth}{0.1mm}}

% width of pdf found box
\newlength\eyja



% #################################################################### COVERS #

\def\dictnameCZ{islandsko-český studijní slovník}
\def\dictnameIS{íslensk-tékknesk stúdentaorðabók}

\newcommand*{\frontcoverimages}{%
  ds_image_blaberjalyng_0_2.jpg,
  ds_image_lundi_0_1.jpg,
  ds_image_hestur_0_1.jpg,
  ds_image_hrafnafifa_0_2.jpg,
  ds_image_baldursbra_0_1.jpg,
  20948.jpg,
  ds_image_hvalur_0_1.jpg,
  ds_image_blodberg_0_1.jpg}

\newcommand*{\backcoverimages}{%
  ds_image_spoi_0_1.jpg,
  ds_image_mosalyng_0_1.jpg,
  ds_image_blaklukkulyng_0_1.jpg,
  20787.jpg,
  ds_image_ufsi_0_2.jpg,
  ds_image_gullbra_0_1.jpg,
  ds_image_islandsfifill_0_2.jpg,
  21137.jpg}

\newcommand\makecoverwith[1]{
 \pagecolor{title}\afterpage\nopagecolor
 \begin{center}
 \vspace*{0.3cm}%{\fill}
 {\Huge\scshape\color{white}%
   \dictnameCZ\\\dictnameIS}
 \vspace*{1em}                        % I think this is necessary
 \begin{figure}[ht]\centering%
   \setlength\fboxsep{0pt}\setlength\fboxrule{0.5pt}%
   \foreach \file in #1{
     \fbox{\includegraphics[width=5.5cm]{\file}}}
 \end{figure}
 \vspace*{\fill}
 \end{center}
}















% ####################################################### HYPERREF & PDF INFO #

% unicode neccessary so that national characters in hypersetup appear ok
\usepackage[pdftex, unicode, hyperfootnotes=false]{hyperref}

% hyperlinks in black
\makeatletter
\let\Hy@linktoc\Hy@linktoc@none
\makeatother

\hypersetup
  { pdftitle={Islandsko-český studijní slovník}
  , pdfauthor={Aleš Chejn, Jón Gíslason, Ján Zaťko}
  , pdfsubject={Islandsko-český studijní slovník}
  , pdfkeywords={Islandsko-český studijní slovník%
               , islandština, čeština, slovník}
  , bookmarks=true
  , colorlinks=true
  , citecolor=black
  , urlcolor=darkgreen
  , linkcolor=black}


\usepackage[final]{microtype}
