% ############################################################ DOCUMENT CLASS #

\documentclass[ 8pt, c5paper, twoside, openright]{extbook}

%\documentclass[10pt, a4paper, twoside, openright]{extbook}
%\documentclass[ 8pt, B5paper, twoside, openright]{extbook} 

\newlength\lead \setlength\lead{9.6pt} % = 1.2 * 8pt

% ########################################################### PAGE GEOMETRIES #

\usepackage{geometry}

% Basic geometry of document
\geometry
  { headsep    =   \lead
  , textwidth  = 42\lead
  , textheight = 60\lead
  , hmarginratio = 2:3
  , vmarginratio = 2:3
  , bindingoffset = 0cm
  , onecolumn
  }

% Geometry of the dictionary section
\newcommand\dictionarygeometry{\newgeometry
  { textwidth  = 42\lead
  , textheight = 60\lead
  , hmarginratio = 2:3
  , vmarginratio = 3:2
  , bindingoffset = 0cm
  , twocolumn
  }}


% ################################################################# LANGUAGES #

\usepackage{polyglossia}

\setmainlanguage{czech}%{icelandic}
\setotherlanguage{icelandic}%{czech}
\setotherlanguage{latin}

% We wrap these up, just in case 
\def\textIS#1{\texticelandic{#1}}
\def\textCS#1{\textczech{#1}}
\def\textLA#1{\textlatin{#1}}


% ################################################################## ENCODING #

%\usepackage[utf8]{inputenc}
\usepackage{newunicodechar}

%
%% Smashing uppercase letters with diacritics
%% ------------------ THIS IS EVIL AND SHOULD NEVER BE DONE -------------------
%% Don't smash anything if we are on the first run to produce the index
%\IfFileExists{tmp/figures.idx}{
%  % Czech uppercase letters with diacritics:
%  \newunicodechar{Á}{\strut\smash{\' A}} % 00C1
%  \newunicodechar{Č}{\strut\smash{\v C}} % 010C
%  \newunicodechar{Ď}{\strut\smash{\v D}} % 010E
%  \newunicodechar{É}{\strut\smash{\' E}} % 00C9
%  \newunicodechar{Ě}{\strut\smash{\v E}} % 011A
%  \newunicodechar{Í}{\strut\smash{\' I}} % 00CD
%  \newunicodechar{Ň}{\strut\smash{\v N}} % 0147
%  \newunicodechar{Ó}{\strut\smash{\' O}} % 00D3
%  \newunicodechar{Ř}{\strut\smash{\v R}} % 0158
%  \newunicodechar{Š}{\strut\smash{\v S}} % 0160
%  \newunicodechar{Ť}{\strut\smash{\v T}} % 0164
%  \newunicodechar{Ú}{\strut\smash{\' U}} % 00DA
%  \newunicodechar{Ů}{\strut\smash{\r U}} % 016E
%  \newunicodechar{Ý}{\strut\smash{\' Y}} % 00DD
%  \newunicodechar{Ž}{\strut\smash{\v Z}} % 017D
%  % Icelandic uppercase letters with diacritics: ÁÉÍÓÚÝ (ÐÞÆ) and
%  \newunicodechar{Ö}{\strut\smash{\" O}} % 00D6
%}{}
%% ------------------ THAT WAS EVIL AND SHOULD NEVER BE DONE ------------------
%
%% Non-breakable space
%\DeclareUnicodeCharacter{00A0}{~}
%
%% Non-breakable hyphen
%\newcommand*{\nobreakhyphen}{\mbox{-}}
%\DeclareUnicodeCharacter{2011}{\nobreakhyphen}

%\newunicodechar{·}{\BeginAccSupp{method=plain,ActualText={}}·\EndAccSupp{}}

%\def\·{·}%
%\catcode`\·=\active
%\def ·{\·}%


% ############################################################ FONTS, SYMBOLS #

\usepackage{fontspec}

\setmainfont{LinLibertine_R}%
  [ Extension = .otf
  , FontFace = {m} {it}{*I}
  , FontFace = {b} {n} {*Z}
  , FontFace = {bx}{n} {*B}
  ]

%\newfontfamily\termFamily[Scale=MatchUppercase]{TeX Gyre Heros}
%\newfontfamily\ipaFamily[Scale=MatchLowercase]{CMU Serif}

\def\bfseries{\fontseries{b}\selectfont}

\def\termFont{\fontseries{bx}\selectfont}
\def\IPAFont{}

\newcommand\textTerm[1]{{\termFont#1}}
\newcommand\textIPA[1]{{\IPAFont#1}}

%\def\phvfamily{} % LEGACY

% Extracted from the package stmaryrd:
\DeclareSymbolFont{stmry}{U}{stmry}{m}{n}
%\SetSymbolFont{stmry}{bold}{U}{stmry}{b}{n}
\DeclareMathSymbol\shortrightarrow\mathrel{stmry}{"01}
\DeclareMathSymbol\shortuparrow\mathrel{stmry}{"02}

% Methinks these are not necessary
%\usepackage{dingbat}
%\usepackage{manfnt}
%\usepackage{latexsym}

\usepackage{pifont}


% #################################################################### LAYOUT #

\def\normalsize{\fontsize{8pt}{1\lead}\selectfont}
\def\large{\fontsize{ 9.600000pt}{2\lead}\selectfont}
\def\Large{\fontsize{11.520000pt}{2\lead}\selectfont}
\def\LARGE{\fontsize{13.824000pt}{2\lead}\selectfont}
\def\huge {\fontsize{16.588800pt}{3\lead}\selectfont}
\def\Huge {\fontsize{19.906560pt}{3\lead}\selectfont}
\def\HUGE {\fontsize{23.887872pt}{3\lead}\selectfont}

% Two columns layout
\setlength\columnsep    {2\lead}
\setlength\columnseprule{0.4pt}

% Necessary for baseline alignment
\topskip=\lead
\raggedbottom
\setlength\parskip{0pt} % it's better to avoid glue

% Temporarily suppress warnings
\hbadness=10000
\vbadness=10000

% This value is ENORMOUS but I can't think of a better non-manual solution.
% Stil, microtype helps a lot to lower it.
\setlength\emergencystretch{17pt}

%\renewcommand\baselinestretch{1.0}
%\setlength\baselineskip{9.6pt}


% ######################################################### ASSORTED PACKAGES #

\usepackage{accsupp}

\makeatletter                                                                                                          
  \def\ACCSUPP@bdc{\special{pdf:code \ACCSUPP@span BDC}}                                                               
  \def\ACCSUPP@emc{\special{pdf:code EMC}}                                                                             
\makeatother

\usepackage[rgb]{xcolor}
\usepackage{xparse}

\usepackage{tikz}
\usetikzlibrary{calc}


%\usepackage{fourier-orns}              % ornaments
%\usepackage{amsmath}                   % non-breakable dash
%\usepackage{hyphenat}                  % no hyphen in abbreviations

\usepackage{hanging}
\usepackage{fix2col}
\usepackage{fixltx2e}                  % subscript
\usepackage{pagecolor}
\usepackage{afterpage}                 % Used in coloring trick for covers
\usepackage{pdfpages}

\usepackage{multirow}
\usepackage{multicol}
\usepackage{rotating}

\usepackage{etoolbox}

\usepackage[ISBN=978-80-260-2385-2,SC0]{ean13isbn}

\usepackage{pgffor}
\usepackage{metalogo}

% ############################################################ BASELINE GRID #

\usepackage{eso-pic}
\usepackage{tikzpagenodes}

% Command to draw a baseline grid
\newcommand\drawbaselinegrid{%
  \begin{tikzpicture}[overlay,remember picture]
    \draw [red!30!white, ultra thin, dashed]
      (0,0) grid [ ystep  = \baselineskip, xstep = \textwidth
                 , shift  = (current page text area.north west)
                 , yshift = -\dp\strutbox
                 ] ++(\textwidth,-\textheight);
    \draw [red!30!white, thin]
      (0,0) grid [ step = \baselineskip, xstep = \textwidth
                 , shift=(current page text area.north)
                 ] ++(0.5\textwidth,-\textheight)
      (0,0) grid [ step = \baselineskip, xstep = \textwidth
                 , shift=(current page text area.north)
                 ] ++(-0.5\textwidth,-\textheight);
    \draw[black!10!white]
      (current page text area.north west)
        rectangle (current page text area.south east);
  \end{tikzpicture}}

% Draw a baseline grid on every page
\AddToShipoutPicture{\drawbaselinegrid}


% ################################################################## METADATA #

% This is not actually being used. ------------------------------------------ !

%\title{%
%  \textbf{Islandsko-český studijní slovník}%
%  \thanks{Tato kniha byla vytvořena v \LaTeX{}u pod Ubuntu 14.04.%
%    Poděkování patří všem autorům, kteří publikují pod svobodnými licencemi.}}
%\author{Aleš Chejn, Ján Zaťko, Jón Gíslason}
%\date{říjen 2011}


% ############################################################## BIBLIOGRAPHY #

\usepackage[style=authoryear]{biblatex}
\addbibresource{slovnik.bib}


% ##################################################################### INDEX #

\usepackage{imakeidx}

% Index of authors of photographs
\makeindex
  [ name = figures
  , title = {Seznam autorů fotografií a ilustrací}
  , intoc
  , columnseprule
  , columnsep = \columnsep
  , program = texindy
  , options = -M mal-icelandic-min
  ]

\def\lettergroupDefault#1{#1}
%\def\lettergroup#1{\bfseries#1}


% ##################################################################### LISTS #

\usepackage{expdlist}
\usepackage{enumitem}

\setlist{nosep, topsep=\baselineskip}
\setlist[itemize]{leftmargin=*,label=\scalebox{.6}{\textbullet}}


% #################################################################### FLOATS #

\usepackage{caption}
\captionsetup{labelformat=empty}

% Alter some LaTeX defaults for better treatment of figures:
% See p.105 of "TeX Unbound" for suggested values. 
% See pp. 199-200 of Lamport's "LaTeX" book for details.
% ------------------------------------- Parameters for ALL pages:
\renewcommand \topfraction    {0.9}   %   max fraction of floats at top
\renewcommand \bottomfraction {0.8}   %   max fraction of floats at bottom
% ------------------------------------- Parameters for TEXT (non float) pages:
\setcounter{topnumber}    {1}
\setcounter{bottomnumber} {1}
\setcounter{totalnumber}  {1}         %   2 may work better
\setcounter{dbltopnumber} {2}         %   for 2-column pages
\renewcommand \dbltopfraction   {0.9} %   fit big float above 2-col. text
\renewcommand \textfraction     {0.07}%   allow minimal text w. figs
% ------------------------------------- Parameters for FLOAT (non text) pages:
\renewcommand \floatpagefraction{0.7} %   require fuller float pages
% N.B.: floatpagefraction MUST be less than topfraction !!
\renewcommand{\dblfloatpagefraction}{0.7} % require fuller float pages
% remember to use [htp] or [htpb] for placement

\makeatletter
\setlength{\@fptop}{0pt}
\setlength{\@fpbot}{0pt plus 1fil}
\makeatother

% This kills spacing around floats
\setlength \intextsep        {0pt}
\setlength \floatsep         {0pt}
\setlength \textfloatsep     {0pt}
\setlength \dbltextfloatsep  {0pt}
\setlength \dblfloatsep      {0pt}

% This kills spacing around captions
\setlength \abovecaptionskip {0pt}
\setlength \belowcaptionskip {0pt}


% #################################################################### TABLES #

\usepackage{booktabs}
\usepackage{tabularx}
\usepackage{longtable}
\usepackage{ltxtable}

\newcommand\LTXfw[1]{\LTXtable{\columnwidth}{#1}}

\LTpre=0pt
\LTpost=0pt

% ------------------------------------------------- I HAVE NO CLUE WHAT THIS IS
\begingroup
  \makeatletter
  \catcode`\-=\active
  \AtBeginDocument{
  \def\@@@cmidrule[#1-#2]#3#4{\global\@cmidla#1\relax
    \global\advance\@cmidla\m@ne
    \ifnum\@cmidla>0\global\let\@gtempa\@cmidrulea\else
    \global\let\@gtempa\@cmidruleb\fi
    \global\@cmidlb#2\relax
    \global\advance\@cmidlb-\@cmidla
    \global\@thisrulewidth=#3
    \@setrulekerning{#4}
    \ifnum\@lastruleclass=\z@\vskip \aboverulesep\fi
    \ifnum0=`{\fi}\@gtempa
    \noalign{\ifnum0=`}\fi\futurenonspacelet\@tempa\@xcmidrule}
  }
\endgroup

% These values ensure that for top, mid and bottom rules
% space above + rule width + space below = baselineskip
\setlength \abovetopsep    {0.43\lead}
\setlength \heavyrulewidth {0.10\lead}
\setlength \belowbottomsep {0.43\lead}
\setlength \aboverulesep   {0.47\lead}
\setlength \lightrulewidth {0.06\lead}
\setlength \belowrulesep   {0.47\lead}

%%% % declention and conjugation tables
%%% \usepackage{floatrow}
%%% \DeclareFloatFont{footnotesize}{\footnotesize}
%%% % "scriptsize" is defined by floatrow, "tiny" not
%%% \floatsetup[table]{font=footnotesize}


% ################################################################### FIGURES #

\usepackage{adjustbox}

\graphicspath{%
  {/home/chejnik/Dokumenty/hvalur.org/images/biolib/full/}%
  {/home/chejnik/Dokumenty/hvalur.org/images/uploaded_files/}}

\newlength\dicfigheight
\newlength\dicfigskip

\newcommand\dicFigure[3]{ % {filename}{caption}{index entry}
  \xdef\dicfigfilename{\ifdummyfigures example-image-a\else#1\fi}
  \setlength\fboxsep{0pt}\setlength\fboxrule{0.5pt}
  \def\dicfigbox{\fbox{%
    \adjincludegraphics [ max height = 0.8\columnwidth
                        , max width  = 0.8\columnwidth ] {\dicfigfilename}}}
  \setlength\abovecaptionskip {\dp\strutbox}
  \setlength\belowcaptionskip {\ht\strutbox} 
  \setlength\dicfigheight {\heightof\dicfigbox+\fboxrule}
  \setlength\dicfigskip
    {\baselineskip*\numexpr1+\dicfigheight/\baselineskip\relax-\dicfigheight}
  \begin{figure}[hbt]
    \centering\vspace*{\dicfigskip}\dicfigbox\caption{#2}\index[figures]{#3}
  \end{figure}
}


% ################################################################## ALPHABET #

\def\alphabet
  {a,á,b,c,d,e,é,f,g,h,i%
  ,í,j,k,l,m,n,o,ó,p,r,s%
  ,t,u,ú,v,w,x,y,ý,þ,æ,ö}

\def\ALPHABET
  {A,Á,B,C,D,E,É,F,G,H,I%
  ,Í,J,K,L,M,N,O,Ó,P,R,S%
  ,T,U,Ú,V,W,X,Y,Ý,Þ,Æ,Ö}

%aábcd(ð)eéfghi
%íjklmnoóp(q)rs
%tuúvwxyýþæö

% ##################################################################### THUMB #

\definecolorset{Hsb}{dicThPalette}{}
  {1,    1, .8, .7  ;2,   28, .8, .7  ;3,   46, .8, .7  ;4,   56, .8, .7
  ;5,   76, .8, .7  ;6,  130, .8, .7  ;7,  153, .8, .7  ;8,  174, .8, .7
  ;9,  196, .8, .7  ;10, 211, .8, .7  ;11, 230, .8, .7}

\def\truncdiv#1#2{((#1-(#2-1)/2)/#2)}
%\def\moduloop#1#2{(#1-\truncdiv{#1}{#2}*#2)}
%\def\modulo#1#2{\number\numexpr\moduloop{#1}{#2}\relax}
\def\modplusop#1#2{(1+#1-\truncdiv{#1}{#2}*#2)}
\def\modplus#1#2{\number\numexpr\modplusop{#1}{#2}\relax}

\newcommand\thColor[1]{dicThPalette\modplus{#1}{11}}

\newcounter{thCurrentNum}
\newcounter{thTotalNum}
\newlength\thHeight
\newlength\thWidth 
\newlength\thMargin
\def\thFont{}

\tikzset{thumb/.style={
  text = white,
  text centered, 
  minimum height = \thHeight,
  minimum width = \thWidth,
  outer sep = 0pt,
  font = \thFont}}

\usepackage{everypage}

\def\thLast{}
\def\thStack{}

\AddEverypageHook{\if\relax\thStack\relax\xdef\thStack{\thLast}\fi}

\newcommand{\thDraw}[3]{%
  \begin{tikzpicture}[remember picture, overlay]
    \node [thumb, fill=\thColor{#1}, anchor=south #2]
       at ($(current page.north #2)-(0,\thMargin+#1\thHeight)$) {#3};
  \end{tikzpicture}}

\newcommand\thDrawRecto[1]{\thDraw{#1}{east}{\csname thLabel#1\endcsname}}
\newcommand\thDrawVerso[1]{\thDraw{#1}{west}{\csname thLabel#1\endcsname}}

\def\thDoRectos{\expandafter\thGobbleWith\thStack<\thDrawRecto>}
\def\thDoVersos{\expandafter\thGobbleWith\thStack<\thDrawVerso>}

\def\thGobbleWith#1,#2<#3>{%
  \if\relax#1\relax\else#3{#1}\fi%
  \if\relax#2\relax\else\thGobbleWith#2<#3>\fi%
  \gdef\thLast{#1,}%
  \gdef\thStack{}}

\def\thPush#1{
  \refstepcounter{thCurrentNum}%
  \expandafter\gdef\csname thLabel\thethCurrentNum\endcsname{#1}%
  \xdef\thStack{\thethCurrentNum,\thStack}}


% Initialize dictionary thumb style
\setcounter{thTotalNum}{33}
\setlength\thHeight {1.75\baselineskip}
\setlength\thWidth  {0.4\paperwidth-0.4\textwidth}
\setlength\thMargin {0.5\paperheight-17\thHeight}
\def\thFont{\sffamily\bfseries\large}


% ############################################################### PAGE STYLES #

\usepackage{fancyhdr}

% -------------------------------------- BASIC PAGE STYLE
\fancypagestyle{plain}{
  \fancyhf\relax                       % Clear header/footer
  \renewcommand \headrulewidth {0.0pt} % No header rule
  \renewcommand \footrulewidth {0.0pt} % No footer rule
  \fancyfoot[C]{\thepage}              % Page number in footer, centred
}

% -------------------------------------- DICTIONARY PAGE STYLE
\fancypagestyle{myheadings}{
  \fancyhf\relax                       % Clear header/footer
  \renewcommand \headrulewidth {0.4pt} % Thin header rule
  \fancyhead[CO]{\thepage\thDoRectos}  % Page number in header, centred,
  \fancyhead[CE]{\thepage\thDoVersos}  %   automatic handling of thumbs
  \fancyhead[LE,LO]{\termFont\rightmark}
  \fancyhead[RE,RO]{\termFont\leftmark}
}

\pagestyle{plain}  


% ############################################################## LOCALIZATION #

% renames the index name
%\addto\captionsczech{%
%  \renewcommand\indexname{{Seznam autorů fotografií a ilustrací}}}

% renames the contents name
%\addto\captionsczech{%
%  \renewcommand\contentsname{Obsah}}

%\renewcommand \contentsname {Obsah}
%\renewcommand \chaptername  {Kapitola}


% ################################################################ SECTIONING #
\usepackage[explicit]{titlesec}

\setlength \beforetitleunit {\lead}
\setlength \aftertitleunit  {\lead}

\titlespacing*{\chapter}   {0em}{*0}{*3} % Remember there's a topskip too
\titlespacing*{\section}   {0em}{*1}{*2}
\titlespacing*{\subsection}{0em}{*1}{*2}

\newcommand\ghostdrop[2]{%
  \raisebox{-#1\lead}[0pt][0pt]{#2}}

\titleformat {\chapter} {} {} {0pt} % {#1}
  {\ghostdrop{1.0}{\parbox[t]{\columnwidth}
     {\LARGE\bfseries\centering\MakeUppercase{#1}}}}

\def\longchapterskip{\vspace{2\lead}}
\def\longsectionskip{\vspace{2\lead}}

\titleformat {\section} {\bfseries} {} {0pt}
  {\ghostdrop{1.0}{\parbox[t]{\columnwidth}
     {\Large\bfseries\centering#1}}}

\titleformat {\subsection} {\bfseries} {} {0pt}
  {\ghostdrop{1.0}{\hspace{1em}\parbox[t]{\columnwidth-1em}
     {\large\bfseries#1}}}


\setlength\multicolsep{0pt}
%\BeforeBeginEnvironment{multicols}{\vspace{-\dp\strutbox}}


% ######################################################## TABLE OF CONTENTS #
\usepackage{titletoc}

\titlecontents {chapter} [0pt] {\addvspace\baselineskip\bfseries}
  {} {} {\titlerule*[2\baselineskip]{}\contentspage}

\titlecontents {section} [\baselineskip] {} {} {}
  {\titlerule*[1\baselineskip]{.}\contentspage}

\setcounter{tocdepth}{1}

% #################################################### ASSORTED CUSTOM MACROS #
\newcommand\blspace[1][1]{\vspace{#1\baselineskip}}


% ####################################################### DICTIONARY COMMANDS #

\definecolor {darkgreen} {rgb} {0.40, 0.01, 0.24}
\definecolor {title}     {RGB} {  16,   13,   32}
\definecolor {newdev}    {RGB} {   3,   11,   99}

\newcommand \dicsymFrequent  {\raisebox{-.2ex}{\color{darkgreen}\ding{167}}}
\newcommand \dicsymSee       {$\shortrightarrow$}
\newcommand \dicsymCompare   {$\shortuparrow$}
\newcommand \dicsymIdiom     {$\diamondsuit$}
\newcommand \dicsymExampleIS {$\triangleright$}
\newcommand \dicsymExampleCS {\guilsinglright}
\newcommand \dicsymProverb   {{\color{newdev}\ding{96}}}
\newcommand \dicsymPhoto
  {\includegraphics[keepaspectratio,height=0.65em]{photo_icon}}

\NewDocumentCommand\dicEntry{o}
  {\par\hangpara{0.25\baselineskip}{1}%
   \IfValueT{#1}{\markboth{#1}{#1}}%
   \ignorespaces}

\newcommand\dicSubEntry
  {\\\hspace*{-\hangindent}}

\NewDocumentCommand\textAlt{mm}
  {\BeginAccSupp{method=pdfstringdef,ActualText={#2}}#1\EndAccSupp{}}

\newcounter{terms}

\NewDocumentCommand\dicMaybeSuper{m}
  {\IfValueT{#1}{\smash{\textsuperscript{#1}}}}
\NewDocumentCommand\dicMaybeSub{m}
  {\IfValueT{#1}{\nobreak\textsubscript{#1}}}

\NewDocumentCommand\dicTerm{smod<>s} % \dicTerm*{term}[te·r|m]<2>
  {\refstepcounter{terms}\label{term:#2\IfValueT{#4}{:#4}}%
   \textIS{\termFont\IfValueTF{#3}{\textAlt{#3}{#2}}{#2}\dicMaybeSuper{#4}}}

\NewDocumentCommand\dicTermDemo{smod<>s} % \dicTerm*{term}[te·r|m]<2>
  {\textIS{\termFont\IfValueTF{#3}{\textAlt{#3}{#2}}{#2}\dicMaybeSuper{#4}}}

\NewDocumentCommand\dicIPA{mo}
  {{\IPAFont[#1]}}

\NewDocumentCommand\dicPos{smo}
  {{\color{darkgreen}\small\textbf{#2}\dicMaybeSub{#3}}}

\NewDocumentCommand\dicFlx{smo}
  {{\color{darkgreen}\footnotesize#2\dicMaybeSub{#3}}}

\NewDocumentCommand\dicExampleCS{m}
  {\dicsymExampleCS~\textit{#1}}
\NewDocumentCommand\dicExampleIS{m}
  {\dicsymExampleIS~\textIS{\textit{#1}}}

\NewDocumentCommand\dicLink{smd<>}
  {\IfBooleanF{#1}{\dicsymSee~}%
   \hyperref[term:#2\IfValueT{#3}{:#3}]{\textIS{\bfseries#2\dicMaybeSuper{#3}}}}
\NewDocumentCommand\dicSynonym{smd<>}
  {(\IfBooleanF{#1}{\dicsymSee~}\textIS{\itshape#2\dicMaybeSuper{#3}\/})}
\NewDocumentCommand\dicAntonym{smd<>}
  {(\IfBooleanF{#1}{\dicsymCompare~}\textIS{\itshape#2\dicMaybeSuper{#3}\/})}

\newcommand\dicPhraseIS[1]{\textIS{\bfseries#1}}
\NewDocumentCommand\dicIdiom{mo}
  {\dicSubEntry{\color{newdev}\textIS{\termFont#1} %
     \IfValueTF{#2}{+ \textIS{\termFont#2} \markboth{#1 + #2}{#1 + #2}}%
                   {\markboth{#1}{#1}}}\dicsymIdiom}
\newcommand\dicProverb
  {\dicSubEntry\dicsymProverb}

\newcommand\dicLangCat[1]{{\footnotesize#1}}
\newcommand\dicFieldCat[1]{{\footnotesize#1}}

\newcommand\dicDirectTranslationCS[1]{#1}
\newcommand\dicIndirectTranslationCS[1]{{\footnotesize#1}}

\usepackage{placeins}

\newcommand*{\dicLetter}[1]{%
  \FloatBarrier\newpage\phantomsection\pdfbookmark[1]{#1}{LETTER\thethCurrentNum}%
  \noindent\parbox[b][9\baselineskip][c]{\columnwidth}
    {\centering\HUGE\strut\smash{\MakeUppercase{#1}~\MakeLowercase{#1}}}%
      %\par\xdef\tpd{\the\prevdepth}
      %\par\prevdepth\tpd%
  \expandafter\thPush{\MakeUppercase{#1}}}

  
%tabulky
\newcommand{\specialcell}[2][l]{%
\begin{tabular}[#1]{@{}l@{}}#2\end{tabular}}

% rule line
\newcommand{\HRule}{\rule{\linewidth}{0.1mm}}


% #################################################################### COVERS #

\def\dictnameCZ{islandsko-český studijní slovník}
\def\dictnameIS{íslensk-tékknesk stúdentaorðabók}

\newcommand*{\frontcoverimages}{%
  ds_image_blaberjalyng_0_2.jpg,
  ds_image_lundi_0_1.jpg,
  ds_image_hestur_0_1.jpg,
  ds_image_hrafnafifa_0_2.jpg,
  ds_image_baldursbra_0_1.jpg,
  20948.jpg,
  ds_image_hvalur_0_1.jpg,
  ds_image_blodberg_0_1.jpg}

\newcommand*{\backcoverimages}{%
  ds_image_spoi_0_1.jpg,
  ds_image_mosalyng_0_1.jpg,
  ds_image_blaklukkulyng_0_1.jpg,
  20787.jpg,
  ds_image_ufsi_0_2.jpg,
  ds_image_gullbra_0_1.jpg,
  ds_image_islandsfifill_0_2.jpg,
  21137.jpg}

\newcommand\makecoverwith[1]{
 \pagecolor{title}\afterpage\nopagecolor
 \begin{center}
 \vspace*{0.3cm}%{\fill}
 {\Huge\scshape\color{white}%
   \dictnameCZ\\\dictnameIS}
 \vspace*{1em}                        % I think this is necessary
 \begin{figure}[ht]\centering%
   \setlength\fboxsep{0pt}\setlength\fboxrule{0.5pt}%
   \foreach \file in #1{
     \fbox{\includegraphics[width=5.5cm]{\file}}}
 \end{figure}
 \vspace*{\fill}
 \end{center}
}


% ####################################################### HYPERREF & PDF INFO #

% unicode neccessary so that national characters in hypersetup appear ok
\usepackage[%pdftex, unicode, 
            hyperfootnotes=false]{hyperref}

% hyperlinks in black
%\makeatletter
%\let\Hy@linktoc\Hy@linktoc@none
%\makeatother

\hypersetup
  { pdftitle={Islandsko-český studijní slovník}
  , pdfauthor={Aleš Chejn, Jón Gíslason, Ján Zaťko}
  , pdfsubject={Islandsko-český studijní slovník}
  , pdfkeywords={Islandsko-český studijní slovník%
                , islandština, čeština, slovník}
  , bookmarks=true
  , colorlinks=true
  , citecolor=black
  , urlcolor=darkgreen
  , linkcolor=black}


% ################################################################# MICROTYPE #

\usepackage[final]{microtype}
